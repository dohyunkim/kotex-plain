%% File `kotexplain.tex`
%%
%% Typesetting UTF-8 hangul on plain e-TeX
%%
%% (C) Copyright 2007-2013 Dohyun Kim <nomos at ktug org>
%%                         Soojin Nam <sjnam at ktug org>
%%
%% This work may be distributed and/or modified under the
%% conditions of the LaTeX Project Public License, either version 1.3c
%% of this license or (at your option) any later version.
%% The latest version of this license is in
%%  http://www.latex-project.org/lppl.txt
%% and version 1.3c or later is part of all distributions of LaTeX
%% version 2006/05/20 or later.
%% ----------------------------------------------------------------
%%
%% changelog:
%% 2013/10/26   2.0.2   xe(lua)texko v2.1
%% 2009/03/06   2.0.1   plaintex wrapper
%% 2009/03/05   2.0.0   do nothing except loading kotexutf.tex
%% 2007/06/26   1.0.5   error if char does not exist
%% 2007/06/26   1.0.4   redefine U+2018, U+2019, U+201C, U+201D
%% 2007/06/25   1.0.3   fix bugs introduced by previous minor upgrade
%% 2007/06/24   1.0.2   lower multiple punctuations.
%% 2007/06/14   1.0.1   \hu was too normal a CS. use \dhucs@hu intead.
%%
\ifx 가가\else
  \input kotexutf
  \expandafter\endinput
\fi

\ifdefined\XeTeXrevision
  \input xetexko.sty
\else
  \ifdefined\directlua
    \input luatexko.sty
  \fi
\fi

\def\hfontname#1{}
\def\hfontsize#1{}
\def\hfont#1#2{}

\endinput

% eof
